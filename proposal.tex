\documentclass{article}
\usepackage{geometry,graphicx,hyperref}

\geometry{letterpaper,margin=1in}
\def\gamename{Gravity Well}
\def\gameti{\textit{\gamename } }
\def\gametin{\textit{\gamename}}

\title{Proposal: \gameti}
\author{Sameer Bibikar, Sneha Pendharkar\\
EE 319K -- Prof. Yerraballi}

\begin{document}
\maketitle
\section{Overview}
\gameti  is a physics-based two-dimensional obstacle course game.
The player controls a spaceship fated to travel through an abnormal asteroid field.
Each asteroid has a remarkable mass density and exerts a significant
gravitational force on the spaceship. Thus, the player must not only navigate the field,
but also predict the gravitational force on the spaceship and adjust the spaceship accordingly.
\section{Basic Rules}
\begin{itemize}
	\item The player begins with three lives, which represent ship hull strength.
	\item If the player collides with an asteroid, one of these lives will be lost.
	\item If the player loses all his lives, the game is over.
	\item Asteroids may come in different sizes, which means different mass and thus
		a different gravitational force exerted on the ship.
	\item The level (``field'') ends when the player passes a certain distance forward.
	\item The player may only control the thrusters on the ship, not the ship's velocity
		directly. This presents a considerable challenge, as the velocity of the
		ship may not always be under the player's control!
\end{itemize}
\section{Features}
\begin{itemize}
	\item \textbf{Dynamic force-based physics engine} -- Realistic motion of the spaceship and asteroids
		is possible.
	\item \textbf{Hull repair toolkits} -- Remnants of ill-fated spaceships which allow 
		the player to repair his own ship.
	\item \textbf{Dynamic level generation} -- Asteroids will be placed automatically by
		the \gameti  system according to a formula incorporating the difficulty.
	\item \textbf{Score calculation} -- More efficient players are rewarded by giving
		bonus points for quickness and hull repair toolkits collected.
	\item (Optional.) \textbf{Dynamic menus} -- The main menus and help menus will use the
		physics engine to display interesting animations such as a two-dimensional
		star system, or relevant images to go along with the help text.
\end{itemize}
\section{Difficulty}
This game could become very difficult to write (and computationally expensive!) easily.
Challenges involved include:
\begin{itemize}
	\item Physics engine based on forces. This is mostly written, but must be incorporated
		into any other code we will write. This should be fairly simple.
	\item Scrolling screen. Since we want to keep the player at the center of the game,
		all objects will be rendered relative to the spaceship instead of moving the
		spaceship around. This will mean that we need to spend time thinking about 
		how the rendering code will look, as the display's origin does not coincide with
		the origin of the game's coordinate system. However, allowing the game's origin
		to be wherever it needs to be means that the physics engine is allowed to be simple,
		and few adjustments need to be made for scaling and translation of the coordinates
		of entities to be rendered.
	\item (Optional.) Dynamic background. Instead of one background image,
		 \gameti  will use ``shooting star'' entities which
		zoom around the background behind the asteroids. These will be sprites of 1x5, so
		it will be trivial to erase the previous position of the entity and redraw the new
		position. With this implementation, the background can also be dynamic, with randomized
		shooting star positions and velocities, allowing for a more immersive experience.
		However, rendering this many entities at once may be too computationally costly
		for our little microcontroller, so this must be evaluated with that in mind.
	\item Speed/frame rate must be constant, but this should be handled by a system timer, which
		increments a counter in the main loop telling it how many more times it has to run to keep up.
		It may be a challenge to keep the physics engine and drawing functions efficient enough
		to properly run on time on the controller.
	\item Music and audio. At the very least, we will have sound effects for menu items and things which
		happen in the game. This will take up some processor time, but the code can be taken from a
		previous lab.
		An interrupt will be used to drive the sound.
	\item Dynamic memory allocation. Since the level generation system will generate a random number of asteroids,
		we cannot know how much memory to allocate beforehand. Thus, each pass, the generator
		will use \verb malloc()  to allocate space for the asteroids. This space must be \verb free() 'd
		later in order to conserve RAM.
	\item Linked list. Since the list of asteroids will dynamically change, we will need to
		provide a robust linked list implementation. 
\end{itemize}

\section{Hardware requirements fulfilled}
\begin{itemize}
	\item \textbf{Buttons} will be used to control menus and pause the game.
	\item The \textbf{potentiometer} will be used to control the spaceship throttle.
	\item The \textbf{display} will be used to display the game.
	\item \textbf{Interrupts} will be used, at the very least, for sound.
	\item \textbf{LEDs} will be used to show health and can flash when the ship is damaged.
		If they are to flash, another interrupt will be used.
	\item The \textbf{DAC} will be used for sound.
\end{itemize}

\section{Implementation decisions}
\subsection{User input}

Users will provide input to \gameti  by pushing buttons and adjusting the potentiometer.
\gameti  will use GPIO interrupts on button press. The ISR will add the pushed button into a event queue,
similar to that of SDL but far simpler. The main thread will poll this queue every tick and perform
corresponding actions as needed.

\subsection{Entities and updating}

All objects on which physics can act will be associated with \verb Entity  structures. An object's
 \verb Entity  keeps track of position, velocity, and mass of the object. This is all the physics engine requires in order to
compute force, velocity and position on objects.

When a collection of objects is not static, the pointers to the objects will be stored in a stack, which will be
implemented with an array. Ideally, this will be of a fixed size per level, as every tick, the objects will be 
popped off the stack to check if they are needed, and new objects could be added. When calculating the 
gravitational force on the player, the stack will be treated as an array and iterated through.

 When old objects
are to be removed, all the objects will be popped off the stack and each checked if they are to be removed. 
Any objects which are not to be deleted will be pushed onto a different stack of the same size. The pointer to the
stack of the objects will be then updated to the new stack, and the pointer which originally pointed to the new
stack will be updated to point to the stack from which the objects were popped.

Since only pointers are stored in the array, and a maximum of around 100 objects will be in the collection,
the RAM usage will be significant ($100 \times 4 = 400$ bytes for the pointers, and $100 \times 20 = 2$ kilobytes
for the objects themselves) but not too much for the 32 kilobytes of RAM on the machine.

Even though 100 objects will not be rendered on the screen at once, it is important to keep sufficient objects on 
each side of the screen in case the user moves left or right. When objects are being created, they will be randomly
dispersed within a range of distances in front of the ship, and a range left and right of the ship.

As for shooting star objects, they can be tracked with a less complex system with a constant velocity in one direction
only and no mass, allowing for a savings of 8 bytes per object. Thus, the RAM cost of having 100 of these objects is
$100 \times 4 = 400$ bytes for the pointers, and $100 \times 12 = 1.2$ kilobytes for the objects themselves. 

One caveat for this implementation, however, is that the \verb malloc()  and \verb free()  provided by
\verb stdlib.h , as well as the garbage collection of old objects which we implement, must be good enough
so as not to fill up the entire RAM with objects we don't need any more.

\subsection{Main loop structure}

In order to differentiate the two types of screens the user will see: menus and the game itself, each time the main loop
runs, it will check a flag and execute the following accordingly:

\begin{enumerate}
\item \textbf {Menu}. An integer will keep track of which menu is shown at a particular time. Within the menu,
a smaller loop will run, rendering/updating the menu screen if needed and polling the event queue. The physics
engine may be used, but the potentiometers will not be polled.
\item \textbf{Game}. Game parameters will be set according to the level number, and ship paramters will be initialized.
Score may also be initialized if a new game was started. Then, the game will enter a \emph{game loop}, which is described
in exhaustive detail in the next subsection. After the game loop is over, the game will check the exit status and enter
a menu depending on this status (e.g. game over, next level, you win).
\end{enumerate}

\subsection{Game loop structure}

This game loop is where the actual game is played. There are a number of steps which must occur every tick:

\begin{enumerate}
\item Poll the event queue and process each event. If the pause button is pressed, the screen will blank out,
the text ``GAME PAUSED'' will display, and the game will wait for user input before doing anything else. (It is still
within the game loop, waiting for input.) When the game is resumed, nothing will be redrawn at this stage because
it will be redrawn later in the tick anyway.
\item Poll the potentiometers and get the ship thruster force.
\item Evaluate gravitational force on the ship. This requires that the game iterate through all asteroids and calculate 
their gravitational force on the ship.
\item Any object onto which forces are exerted must have its velocity updated using the found net force. So far, the
only object to which this will happen is the player's spaceship.
\item Any object which moved relative to the screen ($\vec{v}_{object} - \vec{v}_{player} \ne \vec0$) must be erased.
\item Update position of all objects whose position need to be updated ($\vec{v} \ne \vec{0}$).
\item Get rid of all objects which are below the screen by some threshold. (necessary to save RAM!)
\item Draw all objects whose position changed on the screen (or redraw all objects if we're not tracking that)
\item Check if the level is over; if so, set the menu flag and number and exit the game loop.
\item Check collisions. If the player collided with a power-up, then increase the player's health. If it was an
asteroid, then decrease the player's health.
\item Check if the player died (health $\le 0$). If so, set the menu flag and number to ``game over'' and exit game loop.
\item Every so often, generate new asteroids in front of the player.
\item Repeat.
\end{enumerate}

\subsection{Sound}

To keep things simple, we will implement a monophonic sound system in \gametin. By default, music will be playing, but
it will be interrupted by sound effects when necessary. This means that we can keep the same system we implemented in the
Piano DAC lab, but augment it with the sound effect system and remove the piano playing part. Two timer interrupts
will be required to implement sound.

\subsection{Graphics}

The spaceship will be shown as a sprite, and asteroids will either be rendered as sprites or circles. The shooting stars
will be rendered as tiny sprites of $5\times1$ pixels.

In order to erase something on the screen, the entire hitbox of the object will be filled black. The object will then be
redrawn in the new position.

A bit of arithmetic is required to get the correct position to which to render the object on the screen. Since our API
for the ST7735 treats $(0,0)$ as the top left corner of the display, we must first get the position of the object relative
to the player's spaceship's top left corner:
$$\vec s_{rel} = \vec s_{obj} - \vec s_{ship}$$
In order for this to make sense, we need to have the physics engine treat the $+y$-direction as downwards onto the screen,
basically the same direction that the ST7735 thinks $+y$ is.
Then, we need to add to that the relative position of the ship on the screen, which should be close to the center of the screen.
$$\vec s_{display} = \vec s_{rel} + \vec s_{shipdisplay}$$

\subsection{Physics}

The physics engine is entirely based on forces. For each object, the engine keeps track of mass, velocity, and position.
Asteroids exert a gravitational force on the spaceship. This force is given by

$$\vec F_g = \frac{Gm_1 m_2}{r^2}\hat r\textrm{,}$$

\noindent where $G$ is a constant, $m_1$ and $m_2$ are the masses of the asteroid and spaceship, and $\vec r$ is the difference in
position between the asteroid and spaceship.

This force will be added to the force exerted on the spaceship by its thrusters, and the acceleration will be calculated 
with the ubiquitous
$$\vec F = m \vec a\textrm{.}$$

Then the velocity and position of the object will be updated as follows:

$$\Delta \vec v = \vec a \Delta t$$
$$\Delta \vec s = \vec v \Delta t$$

Note that the velocity and position of the ship are always given by the computer! While the player may have some control
over the acceleration, if moving too quickly into an asteroid, there may be no escape.

\end{document}

